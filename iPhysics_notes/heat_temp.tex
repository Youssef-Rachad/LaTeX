\documentclass{article}
\usepackage[utf8]{inputenc}

\title{iPhysics Heat \& Temperature}
\author{y}
\date{April 2021}

\begin{document}

\maketitle
%\tableofcontents
\section{Temperature:}

\paragraph{paragraph}
Temperature is a measure of how hot something is in Kelvin.
It is measured with a thermometer.
Thermal equilibrium is the state where $T^0$ of the objects, A and B, is the same.
Converting between Celsius degrees and Kelvin degrees follow the relation:
$$T_k = T_c + 273$$

\section{Thermal Expansion}
\paragraph{paragraph}
Change in dimensions of a body following a change in its temperature.

\begin{tabular}{ |c|c|c| }
\hline
    linear: & $l = l_0\ (\alpha\Delta T + 1)$ & $\alpha$ coeff. of linear expansion\\
    \hline
    superficial: & $A = A_0\ (\gamma\Delta T + 1)$ & $\gamma = 2\cdot\alpha$ coeff. of superficial expansion\\
    \hline
    volume: & $V = V_0\ (\beta\Delta T + 1)$ & $\beta = 3\cdot\alpha$ coeff. of linear expansion\\
    \hline
\end{tabular}

\paragraph{}
Solids undergo all three types of expansion, while liquids and gasses only undergo volume expansion.
\section{Heat}
Transfer of energy from body of higher $T^0$ to lower $T^0$

Heat Capacity is the quantity of heaat required to increase the temperature of a body by 1K. $$Q = mc\Delta T\  [J\cdot ^0K^-1]$$

Specific HEat Capacity (c): energy required to increase the temperature of a 1kg of a given substance by 1K, in [$\frac{J}{kg\cdot ^0K}$].

3 Methods of heat transfer: 
\begin{enumerate}
    \item Conduction
    \item Convection
    \item Radiation
\end{enumerate}
\end{document}
