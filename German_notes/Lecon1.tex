\documentclass[12pt]{article}
\usepackage[margin=1in]{geometry}
\usepackage[all]{xy}

\usepackage{geometry}        
\geometry{letterpaper}    
\usepackage{graphicx}


\begin{document}
\noindent Allemand \hfill Leçon 1 
\hrule
\section{Majuscules aux noms}
Tous les noms (propres et communs) commencent avec une lettre majuscule. 
\textit{ex: Wasser, Bröt, Kind.}
\section{Genre}
Il y a 3 genres: 
\begin{enumerate}
    \item Masculin
    \item Féminin
    \item Neutre
\end{enumerate}
Il y a aussi quelques règles pour identifier un nom:
\begin{enumerate}
    \item Tous les noms en suffixe -us, -ig,
    la plupart en -er sont masculin. \\
    \textit{ex: der Fernseher (le téléviseur), der Optimismus, der Honig (le miel)}.
    \item Tous les noms en suffixe -ung, -eit, -schaft sont féminin.\\
    \textit{ex: die Übung (l'exercice), die Freiheit (la liberté), die Freundschaft (l'amitié)}
    \item Tous les noms en -tum, -icht, -chen sont neutres.\\
    \textit{das Eigentum (la propriété), das Gesicht (le visage), das Mädchen (la fille)}
\end{enumerate}
\section{Umlaut}
Umlaut est une  transformation du son  et modifie le timbre d'une voyelle.
\section{Conjugaison}
Le verbe boire - \textit{trinken}
\begin{center}
\begin{tabular}{|c|c|}
        \hline
        ich trink\textbf{e} & wir trink\textbf{en}  \\
        \hline
	    du trink\textbf{st} & ihr trink\textbf{t} \\
        \hline
	    er/sie/es trink\textbf{t} & sie/Sie trink\textbf{en}\\
        \hline
    \end{tabular}
\end{center}
Sie est la formule de politesse de la $2^e$ personne du pluriel.\\\\
\fbox{\begin{minipage}{25em}
Exemples:\\
Ich trinke eine Apfelsaft.\\
Du trinkst Honig.
\end{minipage}}
\end{document}
